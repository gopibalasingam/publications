\documentclass[12pt,letterpaper]{exam}  
\usepackage[utf8]{inputenc}  
\usepackage[T1]{fontenc}  
\usepackage{times}  
\usepackage[left=1in,right=1in,top=1in,bottom=1in]{geometry}  
\usepackage{amsmath}  
  
\pagestyle{headandfoot}  
\firstpageheader{}%  
  {SNC1W – Grade 9 Science Final Exam (Harder)}%  
  {June 2025}  
\firstpagefooter{Permitted materials: Non‐programmable scientific calculator; ruler; pencils; erasers; 2 reference sheets}{Page \thepage\ of \numpages}{Time: 2 hours}  
\runningfooter{}{}{}  
  
\begin{document}  
  
\begin{center}  
  \large\bfseries  
  Name: \underline{\hspace{2.5in}} \quad Period: \underline{\hspace{0.5in}} \quad Date: \underline{\hspace{1in}}  
\end{center}  
  
\vspace{0.2in}  
\noindent\textbf{INSTRUCTIONS}  
\begin{itemize}  
  \item This exam has TWO sections. Write all answers on this paper.  
  \item Section A: Multiple Choice (25 marks; suggested time 30 min).  
  \item Section B: Extended Response (55 marks; suggested time 90 min).  
  \item Show all work, label all diagrams, and answer in complete sentences when required.  
  \item Total marks: 80.  
\end{itemize}  
  
\renewcommand{\questionlabel}{\thequestion.}  
\renewcommand{\choiceshook}{\setlength{\itemsep}{6pt}}  
  
\section*{SECTION A: MULTIPLE CHOICE (25 marks)}  
  
\begin{questions}  
  
% Unit 0  
\question[1] A value measured as 0.0060 m has how many significant figures?    
\begin{oneparchoices}  
  \choice 1  
  \choice 2  
  \choice 3  
  \choice 4  
\end{oneparchoices}  
  
\question[1] A scale reads 12.430 g with an uncertainty of ±0.01 g. The measurement rounded to the correct number of significant figures is:    
\begin{oneparchoices}  
  \choice 12.43 g  
  \choice 12.4 g  
  \choice 12.430 g  
  \choice 12.44 g  
\end{oneparchoices}  
  
\question[1] Convert 2.75 h to seconds.    
\begin{oneparchoices}  
  \choice 9 900 s  
  \choice 2 750 s  
  \choice $9.6\times10^3$ s  
  \choice $1.65\times10^4$ s  
\end{oneparchoices}  
  
\question[1] Express 0.000089 in scientific notation.    
\begin{oneparchoices}  
  \choice $8.9\times10^{-4}$  
  \choice $8.9\times10^{-5}$  
  \choice $8.9\times10^{4}$  
  \choice $8.9\times10^{5}$  
\end{oneparchoices}  
  
\question[1] Which term describes a data point that lies well outside the general trend of a data set?    
\begin{oneparchoices}  
  \choice median  
  \choice outlier  
  \choice mode  
  \choice anomaly  
\end{oneparchoices}  
  
% Unit 3  
\question[1] Two resistors, 10 Ω and 20 Ω, are connected in parallel. Their equivalent resistance is:    
\begin{oneparchoices}  
  \choice 6.7 Ω  
  \choice 30 Ω  
  \choice 15 Ω  
  \choice 2.0 Ω  
\end{oneparchoices}  
  
\question[1] In a circuit, the voltage drop across a 6 Ω resistor carrying 2.0 A is:    
\begin{oneparchoices}  
  \choice 3.0 V  
  \choice 6.0 V  
  \choice 12 V  
  \choice 0 V  
\end{oneparchoices}  
  
\question[1] Which instrument must be inserted in \emph{series} with a component to measure the current through it?    
\begin{oneparchoices}  
  \choice voltmeter  
  \choice ammeter  
  \choice ohmmeter  
  \choice galvanometer  
\end{oneparchoices}  
  
\question[1] A 15 Ω resistor carries 3.0 A. The power dissipated by this resistor is:    
\begin{oneparchoices}  
  \choice 45 W  
  \choice 135 W  
  \choice 5.0 W  
  \choice 180 W  
\end{oneparchoices}  
  
\question[1] Three identical 1.5 V cells (negligible internal resistance) are connected in series. The total voltage is:    
\begin{oneparchoices}  
  \choice 1.5 V  
  \choice 3.0 V  
  \choice 4.5 V  
  \choice 0 V  
\end{oneparchoices}  
  
% Unit 2  
\question[1] Which of these is a heterogeneous mixture?    
\begin{oneparchoices}  
  \choice saline solution  
  \choice air  
  \choice granite  
  \choice ethanol  
\end{oneparchoices}  
  
\question[1] An element has atomic number 14 and mass number 28. The number of neutrons in its nucleus is:    
\begin{oneparchoices}  
  \choice 14  
  \choice 28  
  \choice 42  
  \choice 7  
\end{oneparchoices}  
  
\question[1] Which compound contains both ionic and covalent bonds?    
\begin{oneparchoices}  
  \choice NaCl  
  \choice CO₂  
  \choice Ca(NO₃)₂  
  \choice CH₄  
\end{oneparchoices}  
  
\question[1] A compound has empirical formula CH₂ and molar mass 56 g/mol. Its molecular formula is:    
\begin{oneparchoices}  
  \choice C₂H₄  
  \choice C₄H₈  
  \choice C₃H₆  
  \choice C₅H₁₀  
\end{oneparchoices}  
  
\question[1] Balance the combustion of ethane:    
\[  
  \text{C}_2\text{H}_6 + \text{O}_2 \;\longrightarrow\; \text{CO}_2 + \text{H}_2\text{O}  
\]  
The smallest whole‐number coefficients (C₂H₆, O₂, CO₂, H₂O) are:    
\begin{oneparchoices}  
  \choice 1,3,1,3  
  \choice 2,7,4,6  
  \choice 1,2,2,3  
  \choice 1,4,2,3  
\end{oneparchoices}  
  
% Unit 1  
\question[1] A keystone species is best described as one that:    
\begin{oneparchoices}  
  \choice produces most energy in an ecosystem  
  \choice controls population sizes through predation  
  \choice has no natural predators  
  \choice decomposes organic matter  
\end{oneparchoices}  
  
\question[1] In an energy pyramid, if producers capture 5 000 kJ/m²·year, the tertiary consumers receive approximately:    
\begin{oneparchoices}  
  \choice 500 kJ/m²  
  \choice 50 kJ/m²  
  \choice 5 kJ/m²  
  \choice 0.5 kJ/m²  
\end{oneparchoices}  
  
\question[1] Which of these is a density‐independent factor limiting population growth?    
\begin{oneparchoices}  
  \choice disease  
  \choice competition  
  \choice fire  
  \choice predation  
\end{oneparchoices}  
  
\question[1] Which process directly increases atmospheric CO₂ concentration?    
\begin{oneparchoices}  
  \choice photosynthesis  
  \choice respiration  
  \choice combustion  
  \choice sedimentation  
\end{oneparchoices}  
  
\question[1] Which process is NOT part of the carbon cycle?    
\begin{oneparchoices}  
  \choice decomposition  
  \choice assimilation  
  \choice nitrification  
  \choice combustion  
\end{oneparchoices}  
  
% Unit 4  
\question[1] A Hertzsprung–Russell (H–R) diagram plots:    
\begin{oneparchoices}  
  \choice luminosity vs temperature  
  \choice mass vs brightness  
  \choice distance vs magnitude  
  \choice age vs size  
\end{oneparchoices}  
  
\question[1] A star of spectral type O is:    
\begin{oneparchoices}  
  \choice cool and dim  
  \choice cool and bright  
  \choice hot and dim  
  \choice hot and bright  
\end{oneparchoices}  
  
\question[1] In the life cycle of a low‐mass star, the stage immediately following the red giant is the:    
\begin{oneparchoices}  
  \choice white dwarf  
  \choice planetary nebula  
  \choice supernova  
  \choice neutron star  
\end{oneparchoices}  
  
\question[1] The observation that spectral lines of distant galaxies shift toward longer wavelengths indicates that the universe is:    
\begin{oneparchoices}  
  \choice contracting  
  \choice static  
  \choice expanding  
  \choice oscillating  
\end{oneparchoices}  
  
\question[1] A meteoroid that survives passage through Earth's atmosphere and lands on the surface is called a:    
\begin{oneparchoices}  
  \choice meteor  
  \choice meteorite  
  \choice asteroid  
  \choice comet  
\end{oneparchoices}  
  
\end{questions}  
  
\newpage  
\section*{SECTION B: EXTENDED RESPONSE (55 marks)}  
  
\begin{questions}  
  
% B1  
\question[11] \textbf{UNIT 0 – STEM SKILLS}  
  
A student determines the density of an irregular rock by water displacement.\\  
Initial water volume: 50.0 mL; Final volume with rock: 63.5 mL; Rock mass: 78.2 g (±0.1 g).  
  
\begin{parts}  
  \part[4] Identify:  
  \begin{subparts}  
    \subpart Independent variable    
    \subpart Dependent variable    
    \subpart Two controlled variables    
  \end{subparts}  
  
  \part[3] Calculate:  
  \begin{subparts}  
    \subpart Volume of the rock (cm³).    
    \subpart Density of the rock (g/cm³).    
  \end{subparts}  
  
  \part[2] Calculate the percent uncertainty in the mass measurement.    
    \\  
    \emph{(percent uncertainty = (absolute uncertainty ÷ measured value) × 100\%)}    
  \part[2] Suggest one improvement to reduce experimental uncertainty.  
\end{parts}  
  
% B2  
\question[11] \textbf{UNIT 2 – CHEMISTRY}  
  
\begin{parts}  
  \part[4.5] Draw Lewis (dot) structures for:\newline  
    \begin{subparts}  
      \subpart NO₃⁻    
      \subpart SO₂    
    \end{subparts}  
    Include all lone pairs and formal charges.  
  
  \part[2] Predict the molecular geometry and approximate bond angle for:\newline  
    \begin{subparts}  
      \subpart NO₃⁻    
      \subart SO₂    
    \end{subparts}  
  
  \part[2] Calculate the percent by mass of oxygen in CaCO₃.  
  
  \part[1] Write the balanced equation for Mg + HCl → H₂ + MgCl₂.  
  
  \part[1.5] A compound is 40.0 % C, 6.7 % H, 53.3 % O by mass. Determine its empirical formula.  
\end{parts}  
  
% B3  
\question[11] \textbf{UNIT 3 – ELECTRICITY}  
  
A 24 V battery is connected to R₁ = 6 Ω in series with a parallel branch of R₂ = 12 Ω and R₃ = 4 Ω.  
  
\[  
  +\,(24\text{V})\;-\;\,R_1\;-\;\bullet\;  
  \begin{cases}  
    R_2\\  
    R_3  
  \end{cases}  
  \;\bullet\;-\;-\quad  
\]  
  
\begin{parts}  
  \part[2] Calculate the equivalent resistance of R₂ ∥ R₃.  
  \part[1] Calculate the total resistance of the circuit.  
  \part[1] Determine the total current from the battery.  
  \part[1] Find the voltage drop across R₁.  
  \part[2] Compute the current through R₂ and R₃.  
  \part[2] Calculate the power dissipated by R₃.  
  \part[2] At \$0.12 per kWh, find the cost to run this circuit for 5.0 h.  
\end{parts}  
  
% B4  
\question[11] \textbf{UNIT 1 – ECOSYSTEMS & CLIMATE CHANGE}  
  
Energy flux (kJ/m²·year): Producers 20 000; Primary 2 000; Secondary 200; Tertiary 20.  
  
\begin{parts}  
  \part[2] Sketch a labelled energy pyramid with these values.  
  \part[3] Calculate transfer efficiencies:  
    \begin{subparts}  
      \subpart Producers → Primary    
      \subpart Primary → Secondary    
      \subpart Secondary → Tertiary    
    \end{subparts}  
  \part[2] Explain TWO reasons why energy transfer between levels is inefficient.  
  \part[2] Name TWO other greenhouse gases (besides CO₂) and one human source each.  
  \part[2] Describe one sustainable practice to reduce CO₂ and how it affects the carbon cycle.  
\end{parts}  
  
% B5  
\question[11] \textbf{UNIT 4 – SPACE EXPLORATION}  
  
\begin{parts}  
  \part[7] List, in order, the seven main stages in the life cycle of a \emph{high‐mass} star.  
  \part[2] Explain, in terms of forces and energy, why fusion in a star's core counteracts gravitational collapse.  
  \part[1] Using Hubble’s law $v=H_0d$ ($H_0=70\,$km/s/Mpc), estimate the recessional velocity of a galaxy at 50 Mpc.  
  \part[1] Name one key observation (besides redshift) that supports the Big Bang theory.  
\end{parts}  
  
\end{questions}  
  
\end{document}  
