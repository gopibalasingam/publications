SNC1W – Grade 9 Science – Final Exam (Revised – Harder)    
June 2025  Time Allowed: 2 hours    
  
Name: ________________________  Period: ___  Date: __________    
  
PERMITTED MATERIALS    
• Non-programmable scientific calculator    
• Ruler, pencils, eraser    
• Two reference sheets (periodic table & formula sheet)    
  
INSTRUCTIONS    
1. This exam has TWO sections. Write all answers on this paper in the spaces provided.    
2. SHOW ALL WORK for calculations; label all diagrams clearly.    
3. TOTAL MARKS: 80 (Section A: 25 marks, Section B: 55 marks)    
  
––––––––––––––––––––––––––––––––––––––––––––––––––––––––––––––––––––––––––––––––––––    
SECTION A: MULTIPLE CHOICE (25 marks; suggested time 30 min)    
Circle the letter of the BEST answer for each. Each question is worth 1 mark.    
  
UNIT 0 – STEM SKILLS    
1. A value measured as 0.0060 m has how many significant figures?    
   A. 1 B. 2 C. 3 D. 4    
2. A scale reads 12.430 g with an uncertainty of ±0.01 g. The measurement rounded to the correct number of significant figures is:    
   A. 12.43 g B. 12.4 g C. 12.430 g D. 12.44 g    
3. Convert 2.75 h to seconds.    
   A. 9 900 s B. 2 750 s C. 9.6 × 10³ s D. 1.65 × 10⁴ s    
4. Express 0.000089 in scientific notation.    
   A. 8.9 × 10⁻⁴ B. 8.9 × 10⁻⁵ C. 8.9 × 10⁴ D. 8.9 × 10⁵    
5. Which term describes a data point that lies well outside the general trend of a data set?    
   A. median B. outlier C. mode D. anomaly    
  
UNIT 3 – CHARACTERISTICS OF ELECTRICITY    
6. Two resistors, 10 Ω and 20 Ω, are connected in parallel. Their equivalent resistance is:    
   A. 6.7 Ω B. 30 Ω C. 15 Ω D. 2.0 Ω    
7. In a circuit, the voltage drop across a 6 Ω resistor carrying 2.0 A is:    
   A. 3.0 V B. 6.0 V C. 12 V D. 0 V    
8. Which instrument must be inserted in series with a component to measure the current through it?    
   A. voltmeter B. ammeter C. ohmmeter D. galvanometer    
9. A 15 Ω resistor carries 3.0 A. The power dissipated by this resistor is:    
   A. 45 W B. 135 W C. 5.0 W D. 180 W    
10. Three identical 1.5 V cells (negligible internal resistance) are connected in series. The total voltage is:    
   A. 1.5 V B. 3.0 V C. 4.5 V D. 0 V    
  
UNIT 2 – CHEMISTRY: NATURE OF MATTER    
11. Which of these is a heterogeneous mixture?    
   A. saline solution B. air C. granite D. ethanol    
12. An element has atomic number 14 and mass number 28. The number of neutrons in its nucleus is:    
   A. 14 B. 28 C. 42 D. 7    
13. Which compound contains both ionic and covalent bonds?    
   A. NaCl B. CO₂ C. Ca(NO₃)₂ D. CH₄    
14. A compound has empirical formula CH₂ and a molar mass of 56 g/mol. Its molecular formula is:    
   A. C₂H₄ B. C₄H₈ C. C₃H₆ D. C₅H₁₀    
15. Balance the combustion of ethane:    
    C₂H₆ + O₂ → CO₂ + H₂O    
   The smallest whole-number coefficients (in order: C₂H₆, O₂, CO₂, H₂O) are:    
   A. 1, 3, 1, 3 B. 2, 7, 4, 6 C. 1, 2, 2, 3 D. 1, 4, 2, 3    
  
UNIT 1 – SUSTAINABLE ECOSYSTEMS & CLIMATE CHANGE    
16. A keystone species is best described as one that:    
   A. produces most energy in an ecosystem    
   B. controls population sizes through predation    
   C. has no natural predators    
   D. decomposes organic matter    
17. In an energy pyramid, if producers capture 5 000 kJ/m²/year, the tertiary consumers will receive approximately:    
   A. 500 kJ/m² B. 50 kJ/m² C. 5 kJ/m² D. 0.5 kJ/m²    
18. Which of these is a density-independent factor limiting population growth?    
   A. disease B. competition C. fire D. predation    
19. Which process directly increases atmospheric CO₂ concentration?    
   A. photosynthesis B. respiration C. combustion D. sedimentation    
20. Which process is NOT part of the carbon cycle?    
   A. decomposition B. assimilation C. nitrification D. combustion    
  
UNIT 4 – SPACE EXPLORATION    
21. A Hertzsprung–Russell (H-R) diagram plots:    
   A. luminosity vs temperature B. mass vs brightness    
   C. distance vs magnitude D. age vs size    
22. A star of spectral type O is:    
   A. cool and dim B. cool and bright    
   C. hot and dim D. hot and bright    
23. In the life cycle of a low-mass star, the stage immediately following the red giant is the:    
   A. white dwarf B. planetary nebula    
   C. supernova D. neutron star    
24. The observation that spectral lines of distant galaxies shift toward longer wavelengths indicates that the universe is:    
   A. contracting B. static C. expanding D. oscillating    
25. A meteoroid that survives passage through Earth’s atmosphere and lands on the surface is called a:    
   A. meteor B. meteorite C. asteroid D. comet    
  
––––––––––––––––––––––––––––––––––––––––––––––––––––––––––––––––––––––––––––––––––––    
SECTION B: EXTENDED RESPONSE (55 marks; suggested time 90 min)    
Answer all FIVE questions. Label each part clearly.    
  
B1. UNIT 0 – STEM SKILLS (11 marks)    
A student determines the density of an irregular rock using water displacement.    
• Initial volume of water in a graduated cylinder: 50.0 mL    
• Final volume with rock submerged: 63.5 mL    
• Mass of rock (digital scale): 78.2 g (±0.1 g)    
  
a) Identify: (4 marks; 1 mark each)    
   i) independent variable ii) dependent variable    
   iii) two controlled variables    
  
b) Calculate: (3 marks; show all work)    
   i) volume of the rock in cm³    
   ii) density of the rock in g/cm³    
  
c) Calculate the percent uncertainty in the mass measurement. (2 marks)    
   [percent uncertainty = (absolute uncertainty ÷ measured value) ×100%]    
  
d) Suggest one improvement to reduce experimental uncertainty. (2 marks)    
  
––––––––––––––––––––––––––––––––––––––––––––––––––––––––––––––––––––––––––––––––––––    
B2. UNIT 2 – CHEMISTRY (11 marks)    
a) Draw the Lewis (electron-dot) structures for: (4.5 marks; 1.5 each)    
   i) NO₃⁻  ii) SO₂     
   Include all lone pairs and any formal charges.    
  
b) Predict the molecular geometry and approximate bond angle for each: (2 marks; 1 each)    
   i) NO₃⁻  ii) SO₂     
  
c) Calculate the percent by mass of oxygen in CaCO₃. (2 marks)    
  
d) Write the balanced chemical equation for the reaction of magnesium metal with hydrochloric acid to produce hydrogen gas and magnesium chloride. (1 mark)    
  
e) A compound is found to contain 40.0% C, 6.7% H and 53.3% O by mass. Determine its empirical formula. (1.5 marks)    
  
––––––––––––––––––––––––––––––––––––––––––––––––––––––––––––––––––––––––––––––––––––    
B3. UNIT 3 – ELECTRICITY (11 marks)    
A 24 V battery is connected to a 6 Ω resistor (R₁) in series with a parallel combination of a 12 Ω resistor (R₂) and a 4 Ω resistor (R₃).    
  
   + ──(24 V)──R₁──●──R₂──●    
                     │      │    
                     └──R₃──┘    
                     │      │    
                     └──────┘    
  
a) Calculate the equivalent resistance of R₂ and R₃ in parallel. (2 marks)    
b) Calculate the total resistance of the circuit. (1 mark)    
c) Determine the total current supplied by the battery. (1 mark)    
d) Calculate the voltage drop across R₁. (1 mark)    
e) Find the current through R₂ and through R₃. (2 marks; 1 each)    
f) Calculate the power dissipated by R₃. (2 marks)    
g) At an electricity cost of \$0.12 per kWh, determine the cost to run this circuit for 5.0 h. (2 marks)    
  
––––––––––––––––––––––––––––––––––––––––––––––––––––––––––––––––––––––––––––––––––––    
B4. UNIT 1 – SUSTAINABLE ECOSYSTEMS & CLIMATE CHANGE (11 marks)    
The following energy flux data represent a temperate forest ecosystem (in kJ/m²/year):    
  • Producers: 20 000    
  • Primary consumers: 2 000    
  • Secondary consumers: 200    
  • Tertiary consumers: 20    
  
a) Sketch a labelled energy pyramid showing these four levels. (2 marks)    
  
b) Calculate the percentage transfer efficiency from: (3 marks; 1 each)    
   i) producers → primary consumers    
   ii) primary → secondary consumers    
   iii) secondary → tertiary consumers    
  
c) Explain TWO reasons why energy transfer between trophic levels is inefficient. (2 marks; 1 each)    
  
d) Identify TWO other major greenhouse gases (besides CO₂) and one anthropogenic source for each. (2 marks; 1 each)    
  
e) Describe one sustainable practice that could reduce atmospheric CO₂, and explain how it affects the carbon cycle. (2 marks)    
  
––––––––––––––––––––––––––––––––––––––––––––––––––––––––––––––––––––––––––––––––––––    
B5. UNIT 4 – SPACE EXPLORATION (11 marks)    
a) List, in order, the seven main stages in the life cycle of a HIGH-MASS star. (7 marks; 1 each)    
  
b) Explain, in terms of forces and energy, why nuclear fusion in a star’s core counteracts gravitational collapse. (2 marks)    
  
c) Using Hubble’s law v = H₀ d (H₀ = 70 km/s/Mpc), estimate the recessional velocity of a galaxy located 50 Mpc away. (1 mark)    
  
d) Name one key observation (other than redshift) that supports the Big Bang theory. (1 mark)    
  
––––––––––––––––––––––––––––––––––––––––––––––––––––––––––––––––––––––––––––––––––––    
